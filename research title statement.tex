
\documentclass[12pt]{article}

\end{document}

\documentclass[12pt]{article}

\usepackage[top=2cm, bottom=2.5cm, left=2cm, right=2cm]{geometry}
\begin{document}
\large{\textbf{ AN IMPLEMENTATION TO MINIMISE THE CONFUSION OF THE SORTING ALGORITHMS USING A JAVASCRIPT\\ } }

\subsection{Problem Statement}
 The increased confusion about the implementation of the different sorting algorithms as it is evident in the different programs while using data structures.
 
 
 Sorting refers to arranging data in a particular format, it can be in ascending or descending order. Sorting algorithm specifies the way to arrange data in a particular order which will simplify searching for results, for example looking for a particular record in database, roll numbers in merit list, a particular telephone number, any particular page in a book.
 
 
 The different \textbf{sorting algorithms} include; \textbf{Bubble Sort, Insertion Sort, Selection Sort, Quick Sort sorts, Merge Sort, Shell Sort.} We shall be working with the following sorting techniques as detailed below:
 
 
 \textbf{Bubble Sort:} Each iteration the largest element in the list moves up towards the last place. Sorting takes place by stepping through all the data items one-by-one in pairs and comparing adjacent data items and swapping each pair that is out of order.
 
 
 \textbf{Insertion Sort:} It is a simple Sorting algorithm which sorts the array by shifting elements one by one.
 
 
 \textbf{Selection Sort} This algorithm first finds the smallest element in the array and exchanges it with the element in the first position, then find the second smallest element and exchange it with the element in the second position and continues in this way until the entire array is sorted.


 \textbf{Quick Sort:} sorts any list very quickly basing on the rule of Divide and Conquer.
 
 
 \textbf{Merge Sort:} also uses divide and conquer. Divides elements in two halves and merges the sorted halves.
 
 
 
 
 \textbf{Shell Sort} this method starts by sorting pairs of elements far apart from each other, then progressively then reduces the gap between elements to be compared.     
\end{document}